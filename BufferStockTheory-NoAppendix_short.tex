\documentclass[BufferStockTheory]{subfiles}
% WARNING: Different execution depending on whether
% 0. Being compiled as standalone document
%    * Compile this file, then main, then this one again
%    * Keep iterating until neither file changes
% 0. Being compiled as subfile of main document
%    * Just compile main document repeatedly

\input{./econtexRoot}\input{\LaTeXInputs/econtex_onlyinsubfile}
\onlyinsubfile{\externaldocument{\LaTeXInputs/BufferStockTheory}} % Get xrefs -- esp to appendix -- from main file; only works properly if main file has already been compiled; 

\onlyinsubfile{\renewcommand{\EqDir}{\econtexRoot/Equations}} 
\begin{document}

% Attempted to make all lines used for Web version contain {Web} (or version with only single curly brace at end) so can be removed with sed 
\providecommand{\versn}{pdf} % Version; like, web or pdf or journal submission 
\ifthenelse{\boolean{Web}}{    % {Web} 
  \renewcommand{\versn}{Web}     % Too hard to figure out passing -output-directory through make4ht through htlatex, so web version is compiled with junk files in main directory 
  \renewcommand{\rootFromOut}{.} % {Web} 
}{}  % {Web} 

% Put tiny info header at top to make it easy track git commit that generates it
\hfill{\tiny \jobname~\versn~\today~{at} \DTMcurrenttime, \input{\ResourcesDir/.git-source-commit}~~\input{\ResourcesDir/.git-public-commit}}

\title{Theoretical Foundations of \\ Buffer Stock Saving}

\author{Christopher D. Carroll\authNum}

\keywords{Precautionary saving, buffer stock saving, marginal propensity to consume, permanent income hypothesis, income fluctuation problem}

\jelclass{D81, D91, E21\\
  \href{https://econ-ark.org}{\includegraphics{\ResourcesDir/PoweredByEconARK}}
}

\renewcommand{\forcedate}{August 20, 2021}\date{\forcedate} % Alternative is \date{\today}
%\renewcommand{\forcedate}{\today}


\maketitle 
\hypertarget{abstract}{}
\begin{abstract}
  This paper builds foundations for rigorous and intuitive understanding of `buffer stock' saving models (\cite{bewleyPIH}-like models that exhibit a wealth target), pairing each theoretical result with quantitative illustrations.  After describing conditions under which a consumption function exists, the paper articulates stricter `Growth Impatience' conditions that guarantee alternative forms of stability -- either at the population level, or for individual consumers.  Together, the numerical tools and analytical results constitute a comprehensive toolkit for understanding buffer stock models. 
\end{abstract}

% Various resources 
\hypertarget{links}{}
\begin{footnotesize}
  \parbox{\textwidth}{
    \begin{center}
      \begin{tabbing}
        \texttt{Dashboard:~} \= \= \texttt{\href{https://econ-ark.org/materials/BufferStockTheory?dashboard}{https://econ-ark.org/materials/BufferStockTheory?dashboard}} \\
        \texttt{~~~~~~PDF:~} \> \> \texttt{\href{https://\owner.github.io/BufferStockTheory/BufferStockTheory.pdf}{https://\owner.github.io/BufferStockTheory/BufferStockTheory.pdf}} \\ % Owner is defined in Resources/owner.tex
        \texttt{~~~Slides:~} \> \> \texttt{\href{https://\owner.github.io/BufferStockTheory/BufferStockTheory-Slides.pdf}{https://\owner.github.io/BufferStockTheory/BufferStockTheory-Slides.pdf}} \\
        \texttt{~~~~~html:~} \> \> \texttt{\href{https://\owner.github.io/BufferStockTheory}{https://\owner.github.io/BufferStockTheory}}    \\
        \texttt{~Appendix:~} \> \> \texttt{\href{https://\owner.github.io/BufferStockTheory\#Appendices}{https://\owner.github.io/BufferStockTheory\#Appendices}}    \\
        \texttt{~~~bibtex:~} \> \> \texttt{\href{https://\owner.github.io/BufferStockTheory/LaTeX/BufferStockTheory-Self.bib}{https://\owner.github.io/BufferStockTheory/LaTeX/BufferStockTheory-Self.bib}}  \\
        \texttt{~~~GitHub:~} \> \> \texttt{\href{https://github.com/\owner/BufferStockTheory}{https://github.com/\owner/BufferStockTheory}} \\
      \end{tabbing}
    \end{center}
    
    A \href{https://econ-ark.org/materials/BufferStockTheory?dashboard}{dashboard} allows users to see the consequences of alternative parametric choices in a live interactive framework; a corresponding \href{https://https://econ-ark.org/materials/BufferStockTheory?launch}{Jupyter Notebook}  uses the \href{https://econ-ark/HARK}{Econ-ARK/HARK} toolkit to produce all of the paper's figures (warning: the notebook may take several minutes to launch).
  } % end parbox{\textwidth}
\end{footnotesize}

\begin{authorsinfo}
  \name{Contact: \href{mailto:ccarroll@jhu.edu}{\texttt{ccarroll@jhu.edu}}, Department of Economics, 590 Wyman Hall, Johns Hopkins University, Baltimore, MD 21218, \url{https://www.econ2.jhu.edu/people/ccarroll}, and National Bureau of Economic Research.}
\end{authorsinfo}

\newcommand{\thankstext}{All figures and numerical results \href{https://econ-ark.github.io/nbreproduce}{can be automatically reproduced} using the \href{https://econ-ark/HARK}{Econ-ARK/HARK} toolkit, which can be cited per our references (\cite{carroll_et_al-proc-scipy-2018}); for reference to the toolkit itself see \href{https://econ-ark.org/acknowledging/}{Acknowleding Econ-ARK}.  Thanks to the \href{https://consumerfinance.gov}{Consumer Financial Protection Bureau} for funding the original creation of the \href{https://econ-ark.org}{Econ-ARK} toolkit; and to the \href{https://sloan.org}{Sloan Foundation} for funding Econ-ARK's \href{https://sloan.org/grant-detail/8071}{extensive further development} that brought it to the point where it could be used for this project.  The toolkit can be cited with its digital object identifier, \href{https://doi.org/10.5281/zenodo.1001067}{10.5281/zenodo.1001067}, as is done in the paper's own references as \cite{carroll_et_al-proc-scipy-2018}.  
  Thanks to Will Du, James Feigenbaum, Joseph Kaboski, Miles Kimball, Qingyin Ma, Misuzu Otsuka, Damiano Sandri, John Stachurski, Adam Szeidl, Alexis Akira Toda, Metin Uyanik, Mateo Vel\'asquez-Giraldo, Weifeng Wu,  Jiaxiong Yao, and Xudong Zheng for comments on earlier versions of this paper, John Boyd for help in applying his weighted contraction mapping theorem, Ryoji  Hiraguchi for extraordinary mathematical insight that improved the  paper greatly, David Zervos for early guidance to the literature, and participants in a seminar at the Johns Hopkins University, a presentation at the 2009 meetings of the Society of Economic Dynamics for their insights, and at a presentation at the Australian National University.}

\ifthenelse{\boolean{Web}}{}{
  \begin{minipage}{\textwidth}
    \tiny \thankstext
\end{minipage}
} % {Web}

\titlepagefinish

% \ifthenelse{\boolean{Web}}{\medskip \noindent {
%     \begin{minipage}{\textwidth}\baselineskip=0.5      \medskip      \tiny       \thankstext \medskip \medskip  \end{minipage} % {Web} 
%   }}{\pagebreak  % {Web}
% } % \end{Web}

\hypertarget{Introduction}{}
\section{Introduction}

\label{sec:intro}

In the presence of realistic transitory and permanent shocks to income \textit{a la} \cite{friedmanATheory} and \cite{muthOptimal}, only one further ingredient is required to construct a microeconomically testable model of consumption: A description of preferences.  Zeldes~\citeyearpar{zeldesStochastic} was the first to calibrate a quantitatively plausible example; his paper spawned a literature showing that such models' predictions can match household life cycle data reasonably well, whether or not explicit liquidity constraints are imposed.\footnote{See \cite{carrollBSLCPIH} or \cite{gpLifeCycle} for arguments that models with only `natural' constraints (see below) match a wide variety of facts; for a model with explicit constraints that produces very similar results, see, e.g., \cite{Cagetti}.}

A connected literature in macroeconomic theory, starting with \cite{bewleyPIH}, has derived limiting properties of related infinite-horizon problems, but only in models more complex than the case with just shocks and preferences.  The extra complexity has been required, in part, because standard contraction mapping theorems (beginning with \cite{bellmanDynamicProgramming} and including those building on Stokey et.~al.~\citeyearpar{slpMethods}) cannot be applied when utility and/or marginal utility are unbounded.  Many proof methods also rule out permanent shocks \textit{a la} \cite{friedmanATheory}, \cite{muthOptimal}, and \cite{zeldesStochastic}.\footnote{See \hyperlink{DiffFromLit}{the fuller discussion} at the end of section \ref{subsec:Setup}.}

This paper's first technical contribution is to articulate conditions under which the infinite-horizon version of the original Friedman-Muth-Zeldes problem (without complications like a consumption floor or liquidity constraints) defines a contraction mapping whose limiting value and consumption functions are nondegenerate as the horizon approaches infinity.  The key condition is a generalization of a condition in \cite{mstIncFluct}, which we call the~\hyperlink{FVAC}{`Finite Value of Autarky Condition'} (the other required condition, the \hyperlink{WRIC}{`Weak Return Impatience Condition'} is unlikely to bind).  Because the proof approach is to construct the infinite limit of finite-horizon recursions, many intermediate results are also useful for the solution of the finite horizon problem.  % Conveniently, the resulting model has analytical properties, like continuous differentiability of the consumption function, that make it easier to work with than many models.  

But the paper's main theoretical contribution is to identify conditions under which `stable' values of the wealth-to-permanent-income ratio exist in the infinite horizon case, either for individual consumers (an individual consumer's wealth can be predicted to move toward a `target' ratio) or for the aggregate (the economy as a whole moves toward a `balanced growth' equilibrium with a constant ratio of aggregate wealth to income).  The requirement for stability is always that the model's parameters satisfy a `Growth Impatience Condition,' whose nature depends on the quantity whose stability is required.  A model that exhibits stability of either kind qualifies as a `buffer stock' model.\footnote{Such models are neither a subset nor a superset of~\cite{bewleyPIH} models.  But closed economies in which capital results from saving and has declining marginal productivity are always `buffer stock' economies under some definition of that term, because capital accumulation causes interest rates to fall, which guarantees that a Growth Impatience Condition will hold in equilibrium (see below).  The more interesting applications are to populations (or economies) whose marginal saving behavior does not determine the relevant interest rate, or in which the marginal product of capital does not fall as capital is accumulated (again, see below).}

\hypertarget{KMP}{} Even without a formal proof of its existence, buffer stock saving has been intuitively understood to underlie central quantitative results in heterogeneous agent macroeconomics; for example, the logic of target saving is central to the claim by \cite{kmpHandbook} in the \textit{Handbook of Macroeconomics} that such models explain why, during the Great Recession, middle-class consumers cut their spending more than the poor or the rich.  The theory below provides the rigorous basis for this claim:  Learning that the future has become more uncertain does not change the urgent imperatives of the poor (their high $\uFunc^{\prime}(\cRat)$ means they -- optimally -- have little room to maneuver).  And, increased labor income uncertainty does not much change the behavior of the rich because it poses little risk to their consumption.  Only people in the middle have both the motivation and the wiggle-room to respond by reducing their spending.

Analytical derivations required for the proofs also provide foundations for many other results familiar from the numerical literature.

The paper's first part begins by defining sufficient conditions for the problem to define a useful (nondegenerate) limiting consumption function (and explains how the model relates to those previously considered in the literature).  The conditions required for convergence are interestingly parallel to those required to ensure a useful solution exists for the \hyperlink{Factors-Defined-And-Compared}{liquidity constrained perfect foresight model}; that parallel is explored and explained.  This analysis establishes limiting properties of the consumption function as resources approach infinity, and as they approach their lower bound; using these limits, the contraction mapping theorem is proven.

The next theoretical contribution is to show that a corresponding model with an `artificial' liquidity constraint (that is, a model that exogenously prohibits borrowing even by consumers who could certainly repay) is a particular limiting case of the model without constraints.  The analytical appeal of the unconstrained model is that it is both mathematically convenient (e.g., the consumption function is twice continuously differentiable), and arbitraily close (cf.\ section \ref{sec:deatonIsLimit}) to less tractable models that have elsewhere been tackled with less convenient methods.  For future authors, the approach here models a strategy of proving interesting propositions in a congenial environment, and then appealing to a limiting argument to establish the analogous proposition in an explicitly constrained but formally more unwieldy environment.
  
 In proving the remaining theorems, the \hyperlink{AnalysisoftheConvergedConsumptionFunction}{next section} examines key properties of the model. First, as \hyperlink{LimitsAsmtToInfty}{cash approaches infinity} the expected growth rate of consumption and the marginal propensity to consume (MPC) converge to their values in the perfect foresight case. Second, as \hyperlink{LimitsAsmtToZero}{cash approaches zero} the expected growth rate of consumption approaches infinity, and the MPC approaches a simple analytical limit.  Next, the central theorems articulate conditions under which different measures of `growth impatience' imply useful conclusions about points of stability (`target' or `balanced growth' points).  

The final section elaborates the conditions under which, even with a fixed aggregate interest rate that differs from the time preference rate, a small open economy populated by buffer stock consumers has a balanced growth equilibrium in which growth rates of consumption, income, and wealth match the exogenous growth rate of permanent income (equivalent, here, to productivity growth). In the terms of \cite{schmitt2003closing}, buffer stock saving is an appealing method of `closing' a small open economy model, because it requires no ad-hoc assumptions.  Not even liquidity constraints.\footnote{The paper's insights are instantiated in the \href{https://econ-ark.org}{Econ-ARK} toolkit, whose \href{https://hark.readthedocs.io/en/stable/reference/ConsumptionSaving/ConsIndShockModel.html}{buffer stock saving module} flags parametric choices under which a problem is degenerate or under which stable ratios of wealth to income may not exist.}


\hypertarget{The-Problem}{}
\section{The Problem}

\subsection{Setup}\label{subsec:Setup}\label{subsec:setup}

The infinite horizon solution is the (limiting) first-period solution to a sequence of finite-horizon problems as the horizon (the last period of life) becomes arbitrarily distant.

That is, for the value function, fixing a terminal date $T$,  we are interested in the term $\vLevBF_{T-n}$ in the sequence of value functions $\{\vLevBF_{T},\vLevBF_{T-1},...,\vLevBF_{T-n}\}$.  We will say that the problem has a `nondegenerate' infinite horizon solution if, corresponding to that value function, as $n \uparrow \infty$ there is a limiting consumption function $\usual{\cFunc}(\mRat) = \lim_{n \uparrow \infty} \cFunc_{T-n}$ which is neither $\usual{\cFunc}(\mRat)=0$ everywhere (for all $\mRat$) nor $\usual{\cFunc}(\mRat)=\infty$ everywhere.

\end{document}

