% -*- enable-local-variables: :all; -*-
\providecommand{\econtexRoot}{}
\renewcommand{\econtexRoot}{..}
\providecommand{\econtexPaths}{}\renewcommand{\econtexPaths}{\econtexRoot/Resources/econtexPaths}
% The \commands below are required to allow sharing of the same base code via Github between TeXLive on a local machine and Overleaf (which is a proxy for "a standard distribution of LaTeX").  This is an ugly solution to the requirement that custom LaTeX packages be accessible, and that Overleaf seems to ignore symbolic links (even if they are relative links to valid locations)
\providecommand{\econtex}{\econtexRoot/Resources/texmf-local/tex/latex/econtex}
\providecommand{\econtexSetup}{\econtexRoot/Resources/texmf-local/tex/latex/econtexSetup}
\providecommand{\econtexShortcuts}{\econtexRoot/Resources/texmf-local/tex/latex/econtexShortcuts}
\providecommand{\econtexBibMake}{\econtexRoot/Resources/texmf-local/tex/latex/econtexBibMake}
\providecommand{\econtexBibStyle}{\econtexRoot/Resources/texmf-local/bibtex/bst/econtex}
\providecommand{\econtexBib}{economics}
\providecommand{\notes}{\econtexRoot/Resources/texmf-local/tex/latex/handout}
\providecommand{\handoutSetup}{\econtexRoot/Resources/texmf-local/tex/latex/handoutSetup}
\providecommand{\handoutShortcuts}{\econtexRoot/Resources/texmf-local/tex/latex/handoutShortcuts}
\providecommand{\handoutBibMake}{\econtexRoot/Resources/texmf-local/tex/latex/handoutBibMake}
\providecommand{\handoutBibStyle}{\econtexRoot/Resources/texmf-local/bibtex/bst/handout}

\providecommand{\FigDir}{\econtexRoot/Figures}
\providecommand{\CodeDir}{\econtexRoot/Code}
\providecommand{\DataDir}{\econtexRoot/Data}
\providecommand{\SlideDir}{\econtexRoot/Slides}
\providecommand{\TableDir}{\econtexRoot/Tables}
\providecommand{\ApndxDir}{\econtexRoot/Appendices}

\providecommand{\ResourcesDir}{\econtexRoot/Resources}
\providecommand{\rootFromOut}{..} % Path back to root directory from output-directory
\providecommand{\LaTeXGenerated}{\econtexRoot/LaTeX} % Put generated files in subdirectory
%\providecommand{\EqDir}{\econtexRoot/Equations} % Put generated files in subdirectory
\providecommand{\EqDir}{Equations} % Put generated files in subdirectory
\providecommand{\econtexPaths}{\econtexRoot/Resources/econtexPaths}
\providecommand{\LaTeXInputs}{\econtexRoot/Resources/LaTeXInputs}
\providecommand{\LtxDir}{LaTeX/}

\documentclass[\econtexRoot/BufferStockTheory]{subfiles}
\providecommand{\econtexRoot}{}
\renewcommand{\econtexRoot}{..}
\providecommand{\econtexPaths}{}\renewcommand{\econtexPaths}{\econtexRoot/Resources/econtexPaths}
% The \commands below are required to allow sharing of the same base code via Github between TeXLive on a local machine and Overleaf (which is a proxy for "a standard distribution of LaTeX").  This is an ugly solution to the requirement that custom LaTeX packages be accessible, and that Overleaf seems to ignore symbolic links (even if they are relative links to valid locations)
\providecommand{\econtex}{\econtexRoot/Resources/texmf-local/tex/latex/econtex}
\providecommand{\econtexSetup}{\econtexRoot/Resources/texmf-local/tex/latex/econtexSetup}
\providecommand{\econtexShortcuts}{\econtexRoot/Resources/texmf-local/tex/latex/econtexShortcuts}
\providecommand{\econtexBibMake}{\econtexRoot/Resources/texmf-local/tex/latex/econtexBibMake}
\providecommand{\econtexBibStyle}{\econtexRoot/Resources/texmf-local/bibtex/bst/econtex}
\providecommand{\econtexBib}{economics}
\providecommand{\notes}{\econtexRoot/Resources/texmf-local/tex/latex/handout}
\providecommand{\handoutSetup}{\econtexRoot/Resources/texmf-local/tex/latex/handoutSetup}
\providecommand{\handoutShortcuts}{\econtexRoot/Resources/texmf-local/tex/latex/handoutShortcuts}
\providecommand{\handoutBibMake}{\econtexRoot/Resources/texmf-local/tex/latex/handoutBibMake}
\providecommand{\handoutBibStyle}{\econtexRoot/Resources/texmf-local/bibtex/bst/handout}

\providecommand{\FigDir}{\econtexRoot/Figures}
\providecommand{\CodeDir}{\econtexRoot/Code}
\providecommand{\DataDir}{\econtexRoot/Data}
\providecommand{\SlideDir}{\econtexRoot/Slides}
\providecommand{\TableDir}{\econtexRoot/Tables}
\providecommand{\ApndxDir}{\econtexRoot/Appendices}

\providecommand{\ResourcesDir}{\econtexRoot/Resources}
\providecommand{\rootFromOut}{..} % Path back to root directory from output-directory
\providecommand{\LaTeXGenerated}{\econtexRoot/LaTeX} % Put generated files in subdirectory
%\providecommand{\EqDir}{\econtexRoot/Equations} % Put generated files in subdirectory
\providecommand{\EqDir}{Equations} % Put generated files in subdirectory
\providecommand{\econtexPaths}{\econtexRoot/Resources/econtexPaths}
\providecommand{\LaTeXInputs}{\econtexRoot/Resources/LaTeXInputs}
\providecommand{\LtxDir}{LaTeX/}

\onlyinsubfile{% https://tex.stackexchange.com/questions/463699/proper-reference-numbers-with-subfiles
    \csname @ifpackageloaded\endcsname{xr-hyper}{%
      \externaldocument{\econtexRoot/BufferStockTheory}% xr-hyper in use; optional argument for url of main.pdf for hyperlinks
    }{%
      \externaldocument{\econtexRoot/BufferStockTheory}% xr in use
    }%
    \renewcommand\labelprefix{}%
    % Initialize the counters via the labels belonging to the main document:
    \setcounter{equation}{\numexpr\getrefnumber{\labelprefix eq:Dummy}\relax}% eq:Dummy is the last number used for an equation in the main text; start counting up from there
}


\onlyinsubfile{\externaldocument{\LaTeXGenerated/BufferStockTheory}} % Get xrefs -- esp to appendix -- from main file; only works properly if main file has already been compiled;

%\renewcommand{\LtxDir}{}
\onlyinsubfile{\renewcommand{\LtxDir}{../LaTeX/}}

%\renewcommand\LineNumber{\the\inputlineno}

\begin{document}
%{\linenumbers}

\hypertarget{ApndxMTargetIsStable}{}
\section{Unique And Stable Target and Steady State Points}\label{sec:ApndxMTargetIsStable}

%\subsection{To Prove}

This appendix proves Theorems~\ref{thm:target} and~\ref{thm:MSSBalExists} and 
\onlyinsubfile{\setcounter{theorem}{1}}

%\begin{theorem}\label{thm:target}
  For the nondegenerate solution to the problem defined in Section~\ref{subsec:Setup} when {\FVAC}, {\WRIC}, and {\GICNrm} all hold, there exists a unique cash-on-hand-to-permanent-income ratio $\mTrg>0$ such that
  \begin{equation}
    \Ex_t [{\mRat}_{t+1}/\mRat_t] = 1 \mbox{~if~} \mRat_t = \mTrg.
    \label{eq:mTarget}
  \end{equation}
  Moreover, $\mTrg$ is a point of `wealth stablity' in the sense that
  \begin{equation}\begin{gathered}\begin{aligned}\label{eq:stability}
    \forall {\mRat}_t\in(0,\mTrg),      \,\,& \Ex_t [{\mRat}_{t+1}] > {\mRat}_t  \\
    \forall {\mRat}_t\in(\mTrg,\infty), \,\,& \Ex_t [{\mRat}_{t+1}] < {\mRat}_t. \notag
  \end{aligned}\end{gathered}\end{equation}
  \end{theorem}


%  \begin{theorem}\label{thm:MSSBalExists}
 For the nondegenerate solution to the problem defined in Section~\ref{subsec:Setup} when {\FVAC}, {\WRIC}, and {\GIC} all hold, there exists a unique pseudo-steady-state cash-on-hand-to-income ratio $\mStE>0$ such that
  \begin{equation}
    \Ex_t [\pShk_{t+1}{\mRat}_{t+1}/\mRat_t] = 1 \mbox{~if~} \mRat_t = \mStE.
    \label{eq:mStE}
  \end{equation}
  Moreover, $\mStE$ is a point of stability in the sense that
  \begin{align}\label{eq:stabilityStE}
    \forall {\mRat}_t\in(0,\mStE),      \,\,& \Ex_{t}[\mLevBF_{t+1}]/\mLevBF_{t} > \PGro \\
    \forall {\mRat}_t\in(\mStE,\infty), \,\,& \Ex_{t}[\mLevBF_{t+1}]/\mLevBF_{t} < \PGro. \notag
  \end{align}
\end{theorem}


  \begin{lemma}
  If both $\check{\mRat}$ and $\hat{\mRat}$ exist, then $\hat{\mRat} < \check{\mRat}$.
  \end{lemma}

  \subsection{Proof of Theorem~\ref{thm:target}}
  
  The elements of the proof of Theorem~\ref{thm:target} are:
\begin{itemize}
\item Existence and continuity of $\Ex_t [{\mRat}_{t+1}/\mRat_t]$
\item Existence of a point where $\Ex_t [{\mRat}_{t+1}/\mRat_t] = 1$
\item $\Ex_t [{\mRat}_{t+1}]-\mRat_{t}$ is monotonically decreasing
\end{itemize}


\subsubsection{Existence and Continuity of
  \texorpdfstring{$\Ex_t [{\mRat}_{t+1}/\mRat_t]$}{Ex-{t}[mRat-{t+1}/mRat-{t}]}}\label{subsubsec:RatExitsCont}
The consumption function exists because we have imposed the sufficient conditions (the $\WRIC$ and $\FVAC$; Theorem~\ref{thm:contmap}).  (Indeed, Appendix~\ref{sec:CIsTwiceDifferentiable} shows that $\cFunc(\mRat)$ is not just continuous, but twice continuously differentiable.)

Section~\ref{sec:cExists} shows that for all $t$, $\aRat_{t-1} = {\mRat}_{t-1} -  \cRat_{t-1} > 0$.  Since ${\mRat}_{t}= {\aRat}_{t-1} \Rnorm_{t} + \tShkAll_{t}$, even if $\tShkAll_{t}$ takes on its minimum value of 0, $\aRat_{t-1} \Rnorm_{t} > 0$, since both $\aRat_{t-1}$ and $\Rnorm_{t}$ are strictly positive.  With $\mRat_{t}$ and $\mRat_{t+1}$ both strictly positive, the ratio $\Ex_t [{\mRat}_{t+1}/\mRat_t]$ inherits continuity (and, for that matter, continuous differentiability) from the consumption function.

\subsubsection{Existence of a point where
  \texorpdfstring{$\Ex_t [{\mRat}_{t+1}/\mRat_t]=1$}
  {Ex-t[mRat-{t+1}/mRat-{t}]=1}.}

Existence of a point where $\Ex_t [{\mRat}_{t+1}/\mRat_t] = 1$ follows from:
\begin{enumerate}
\item Existence and continuity of $\Ex_t [{\mRat}_{t+1}/\mRat_t]$ (just proven)
  \item Existence a point where $\Ex_t [{\mRat}_{t+1}/\mRat_t] < 1$
  \item Existence a point where $\Ex_t [{\mRat}_{t+1}/\mRat_t] > 1$
    \item The Intermediate Value Theorem
    \end{enumerate}

\noindent    \textit{Existence of a point where $\Ex_t [{\mRat}_{t+1}/\mRat_t] < 1$}.
    
\textbf{If {\RIC} holds.}  Logic exactly parallel to that of Section~\ref{subsec:LimitsAsmtToInfty} leading to equation~\eqref{eq:xtp1toinfty}, but dropping the $\PGro_{t+1}$ from the RHS, establishes that
\begin{align}
  \lim_{\mRat_{t} \uparrow \infty} \Ex_{t}[{\mRat}_{t+1}/\mRat_{t}]  & =   
                                                                       \lim_{\mRat_{t} \uparrow \infty} 
                                                                       \Ex_{t}\left[\frac{{\mathcal{\mathcal{R}}}_{t+1}(\mRat_{t}-\cFunc(\mRat_{t}))+{\tShkAll}_{t+1}}{\mRat_{t}}\right] \notag 
  \\  & = \Ex_{t}[(\Rfree/{\PGro}_{t+1})\PatR]  \notag
  \\  & = \Ex_{t}[{\Pat}/{\PGro}_{t+1}]  \label{eq:emgro}
  \\  & < 1 \notag
\end{align}
where the inequality reflects imposition of the \GICNrm~\eqref{eq:GICNrm}.

\textbf{If {\RIC} fails.}  When the {\RIC} fails, the fact that $\lim_{\mRat^{\uparrow} \infty} \cFunc^{\prime}(\mRat) = 0$ (see equation~\eqref{eq:MinMPCInv}) means that the limit of the RHS of~\eqref{eq:emgro} as $\mRat \uparrow \infty$ is $\bar{\Rnorm}=\Ex_{t}[\Rnorm_{t+1}]$.  In the next step of this proof, we will prove that the combination {\GICNrm} and \cncl{\RIC} implies $\bar{\Rnorm} < 1$.

So we have $\lim_{\mRat \uparrow \infty} \Ex_{t}[\mRat_{t+1}/\mRat_{t}] < 1$ whether the {\RIC} holds or fails.

\medskip\medskip

\noindent \textit{Existence of a point where \texorpdfstring{$\Ex_t [{\mRat}_{t+1}/\mRat_t] > 1$}{E[mRat-{t+1}/mRat-{t}] > 1}}.

Paralleling the logic for $\cRat$ in Section~\ref{subsec:LimitsAsmtToZero}: the ratio of $\Ex_{t}[\mRat_{t+1}]$ to $\mRat_{t}$ is unbounded above as $\mRat_{t} \downarrow 0$ because $\lim_{\mRat_{t}\downarrow 0} \Ex_{t}[\mRat_{t+1}] > 0$.

\medskip\medskip

\noindent \textit{Intermediate Value Theorem}.  If $\Ex_t [{\mRat}_{t+1}/\mRat_t]$ is continuous, and takes on values above and below 1, there must be at least one point at which it is equal to one.

\subsubsection{\texorpdfstring{$\Ex_t [{\mRat}_{t+1}] -\mRat_t$}{Delta m} is monotonically decreasing.}

Now define \providecommand{\difFunc}{\pmb{\zeta}} $\difFunc(\mRat_t) \equiv 
\Ex_t[\mRat_{t+1}] - \mRat_t$ and note that
\begin{align}\label{eq:difRatioEquiv}
  \difFunc(\mRat_t) < 0 &\leftrightarrow \Ex_t[{\mRat}_{t+1}/\mRat_t] < 1 
                          \nonumber\\
  \difFunc(\mRat_t) = 0 &\leftrightarrow \Ex_t[{\mRat}_{t+1}/\mRat_t] = 1\\
  \difFunc(\mRat_t) > 0 &\leftrightarrow \Ex_t[{\mRat}_{t+1}/\mRat_t] > 
                          1,\nonumber
\end{align}
so that $\difFunc(\mTrg)=0$. Our goal is to prove that $\difFunc(\bullet)$ is strictly 
decreasing on $(0,\infty)$ using the fact that
\begin{align}
  \difFunc^{\prime}(\mRat_{t}) \equiv  \left( \frac{d}{d\mRat_{t}}\right) \difFunc(\mRat_t)  & = \Ex_{t}\left[
                                                                                               \left( \frac{d}{d\mRat_{t}}\right) \left( 
                                                                                               {\Rnorm}_{t+1}(\mRat_{t}-\cFunc(\mRat_{t}))+%
                                                                                               {\tShkAll}_{t+1} - {\mRat}_t\right) \right] \label{eq:difFuncmRatDecreases} \\
                                                                                             & = \bar{\Rnorm}\left(1-\cFunc^{\prime}({\mRat}_t)\right) - 1.  \notag
\end{align}

% Note that the statement of theorem~\ref{thm:target} did not require the {\RIC} to hold.
Now, we show that (given our other assumptions) $\difFunc^{\prime}(\mRat)$ is decreasing (but for different reasons) whether the {\RIC} holds or fails.

\textbf{If {\RIC} holds}. Equation~\eqref{eq:MinMPCDef} indicates that if the {\RIC} holds, then $\MinMPC >0$.  We show at the bottom of Section~\ref{sec:WRIC} that if the {\RIC} holds then $0 < \MinMPC < \cFunc^{\prime}(\mRat_{t}) < 1$ so that 
\begin{align*}
  \bar{\Rnorm}\left(1-\cFunc^{\prime}({\mRat}_t)\right) - 1 & <  \bar{\Rnorm}(1-\underbrace{(1-\PatR)}_{\MinMPC}) - 1  \\
                                                            & = \bar{\Rnorm}\PatR - 1 \\
                                                            & = \Ex_{t}\left[\frac{\Rfree}{\PGro \pShk}\frac{\Pat}{\Rfree}\right] - 1 \\
                                                            & = \underbrace{\Ex_{t}\left[\frac{\Pat}{\PGro \pShk}\right]}_{= \PatPGroAdj} - 1 
\end{align*}
which is negative because the {\GICNrm} says $\PatPGroAdj < 1$.  

\textbf{If {\RIC} fails.}
Under \cncl{\RIC}, recall that $\lim_{\mRat \uparrow \infty} \cFunc^{\prime}(\mRat) = 0$.  Concavity of the consumption function means that $\cFunc^{\prime}$ is a decreasing function, so everywhere 
\begin{align*}
  \bar{\Rnorm}\left(1-\cFunc^{\prime}({\mRat}_t)\right) & < \bar{\Rnorm}
\end{align*}
which means that $\difFunc^{\prime}(\mRat_{t})$ from~\eqref{eq:difFuncmRatDecreases} is guaranteed to be negative if
\begin{align}
  \bar{\Rnorm} \equiv \Ex_{t}\left[\frac{\Rfree}{\PGro \pShk}\right] & < 1  \label{eq:RbarBelowOne}.
\end{align}
But the combination of the {\GICNrm} holding and the {\RIC} failing can be written:
\begin{align*}
  \overbrace{\Ex_{t}\left[\frac{\Pat}{\PGro \pShk}\right]}^{\PatPGroAdj} & < 1 < \overbrace{\frac{\Pat}{\Rfree}}^{{\PatR}},
\end{align*}
and multiplying all three elements by $\Rfree/\Pat$ gives 
\begin{align*}
  \Ex_{t}\left[\frac{\Rfree}{\PGro \pShk}\right] & < \Rfree/\Pat < 1
\end{align*}
which satisfies our requirement in~\eqref{eq:RbarBelowOne}.

  
\subsection{Proof of Theorem~\ref{thm:MSSBalExists}}

The elements of the proof are:
\begin{itemize}
\item Existence and continuity of $\Ex_{t}[\pShk_{t+1}\mRat_{t+1}/\mRat_t]$
\item Existence of a point where $\Ex_{t}[\pShk_{t+1}\mRat_{t+1}/\mRat_t] = 1$
\item $\Ex_{t}[\pShk_{t+1}\mRat_{t+1}-\mRat_{t}]$ is monotonically decreasing
\end{itemize}

\subsubsection{Existence and Continuity of The Ratio}% of $\hat{\mRat}_{t+1}/\mRat_{t}$.}

Since by assumption $ 0 < \ushort{\pShk} \leq \pShk_{t+1} \leq \bar{\pShk} < \infty$, our proof in~\ref{subsubsec:RatExitsCont} that demonstrated existence and continuity of $\Ex_{t}[\mRat_{t+1}/\mRat_{t}]$ implies existence and continuity of $\Ex_{t}[\pShk_{t+1}\mRat_{t+1}/\mRat_{t}]$.

\subsubsection{Existence of a stable point}

Since by assumption $ 0 < \ushort{\pShk} \leq \pShk_{t+1} \leq \bar{\pShk} < \infty$, our proof in Subsection~\ref{subsubsec:RatExitsCont} that the ratio of $\Ex_{t}[\mRat_{t+1}]$ to $\mRat_{t}$ is unbounded as $\mRat_{t} \downarrow 0$ implies that the ratio $\Ex_{t}[\pShk_{t+1}\mRat_{t+1}]$ to $\mRat_{t}$ is unbounded as $\mRat_{t} \downarrow 0$.

The limit of the expected ratio as $\mRat_{t}$ goes to infinity is most easily calculated by modifying the steps for the prior theorem explicitly:
\begin{align}
  \lim_{\mRat_{t} \uparrow \infty} \Ex_{t}[\pShk_{t+1}\mRat_{t+1}/\mRat_{t}]  & =   
                                                                  \lim_{\mRat_{t} \uparrow \infty} 
                                                                  \Ex_{t}\left[\frac{\PGro_{t+1}\left((\Rfree/\PGro_{t+1})\aFunc(\mRat_{t})+{\tShkAll}_{t+1}\right)/\PGro}{\mRat_{t}}\right] \notag 
  \\   & =   \lim_{\mRat_{t} \uparrow \infty} \Ex_{t}\left[
         \frac{(\Rfree/\PGro)\aFunc(\mRat_{t})+\pShk_{t+1}{\tShkAll}_{t+1}}{\mRat_{t}}
         \right] \notag 
  \\   & =   \lim_{\mRat_{t} \uparrow \infty} \left[
         \frac{(\Rfree/\PGro)\aFunc(\mRat_{t})+1}{\mRat_{t}}
         \right] \notag 
  \\  & = (\Rfree/\PGro)\PatR \label{eq:emgro2}
  \\  & = \PatPGro \notag
  \\  & < 1 \notag
\end{align}
where the last two lines are merely a restatement of the \GICRaw~\eqref{eq:GICRaw}.

The Intermediate Value Theorem says that if $\Ex_{t}[\pShk_{t+1}\mRat_{t+1}/\mRat_t]$ is continuous, and takes on values above and below 1, there must be at least one point at which it is equal to one.

\subsubsection{\texorpdfstring{$\Ex_{t}[\pShk_{t+1}\mRat_{t+1}] -\mRat_t$}{pShk mRat-{t+1} --- mRat-{t}} is monotonically decreasing.}

Define \providecommand{\difFunc}{\pmb{\zeta}} $\difFunc(\mRat_t) \equiv 
\Ex_t[\pShk_{t+1}\mRat_{t+1}] - \mRat_t$ and note that
\begin{align}\label{eq:difLevEquiv}
  \difFunc(\mRat_t) < 0 &\leftrightarrow \Ex_t[\pShk_{t+1}\mRat_{t+1}/\mRat_{t}] < 1 
                          \nonumber\\
  \difFunc(\mRat_t) = 0 &\leftrightarrow \Ex_t[\pShk_{t+1}\mRat_{t+1}/\mRat_{t}] = 1\\
  \difFunc(\mRat_t) > 0 &\leftrightarrow \Ex_t[\pShk_{t+1}\mRat_{t+1}/\mRat_{t}] > 
                          1,\nonumber
\end{align}
so that $\difFunc(\mTrg)=0$. Our goal is to prove that $\difFunc(\bullet)$ is strictly 
decreasing on $(0,\infty)$ using the fact that
\begin{align}
  \difFunc^{\prime}(\mRat_{t}) \equiv  \left( \frac{d}{d\mRat_{t}}\right) \difFunc(\mRat_t)  & = \Ex_{t}\left[
                                                                                               \left( \frac{d}{d\mRat_{t}}\right) \left( 
                                                                                               {\Rnorm}(\mRat_{t}-\cFunc(\mRat_{t}))+%
                                                                                               {\pShk}_{t+1}{\tShkAll}_{t+1} - {\mRat}_t\right) \right] \label{eq:difFuncDecreases} \\
                                                                                             & = (\Rfree/\PGro)\left(1-\cFunc^{\prime}({\mRat}_t)\right) - 1.  \notag
\end{align}

Now, we show that (given our other assumptions) $\difFunc^{\prime}(\mRat)$ is decreasing (but for different reasons) whether the {\RIC} holds or fails (\cncl{\RIC}).

\textbf{If {\RIC} holds}. Equation~\eqref{eq:MinMPCDef} indicates that if the {\RIC} holds, then $\MinMPC >0$.  We show at the bottom of Section~\ref{sec:WRIC} that if the {\RIC} holds then $0 < \MinMPC < \cFunc^{\prime}(\mRat_{t}) < 1$ so that 
\begin{align*}
  \Rnorm\left(1-\cFunc^{\prime}({\mRat}_t)\right) - 1 & <  \Rnorm(1-\underbrace{(1-\PatR)}_{\MinMPC}) - 1  \\
                                                      & = (\Rfree/\PGro)\PatR - 1 
\end{align*}
which is negative because the {\GICRaw} says $\PatPGro < 1$.  

\textbf{If {\RIC} fails.}
Under \cncl{\RIC}, recall that $\lim_{\mRat \uparrow \infty} \cFunc^{\prime}(\mRat) = 0$.  Concavity of the consumption function means that $\cFunc^{\prime}$ is a decreasing function, so everywhere 
\begin{align*}
  \Rnorm\left(1-\cFunc^{\prime}({\mRat}_t)\right) & < \Rnorm
\end{align*}
which means that $\difFunc^{\prime}(\mRat_{t})$ from~\eqref{eq:difFuncDecreases} is guaranteed to be negative if
\begin{align}
  \Rnorm \equiv (\Rfree/\PGro) & < 1  \label{eq:FHWCFails}.
\end{align}

But we showed in Section~\ref{subsec:UncertaintyModifiedConditions} that the only circumstances under which the problem has a nondegenerate solution while the {\RIC} fails were ones where the {\FHWC} also fails (that is,~\eqref{eq:FHWCFails} holds).

\subsection{Proof of Lemma}

\begin{comment}
\subsection{Comment}


Due to the model's nonlinearities the values of $\mRat$ at which the expected growth rate of $\cLevBF$ matches $\PGro$ is very slightly different from the $\mRat$ at which the growth rate at which expected growth of $\mLevBF$ is $\PGro$.  Defining $\grave{\mRat}$ as the $\mRat$ at which $\Ex_{t}[\cRatBF_{t+1}/\cRatBF_{t}]=\PGro$, we can show that to first order $\grave{\mRat} \approx \check{\mRat}.$
\begin{align*}
  \Ex_{t}[\cFunc(\mRat_{t+1})\pShk_{t+1}] & = \cFunc(\mRat_{t}) \label{eq:balgrostableC}.
%  \\ \Ex_{t}[\cFunc(\mRat_{t+1})\pShk_{t+1}] & = \cFunc(\mRat_{t}) 
  \\ \Ex_{t}[\left(\cFunc(\grave{\mRat})+\cFunc^{\prime}(\grave{\mRat})(\mRat_{t+1}-\grave{\mRat})\right)\pShk_{t+1}] & \approx \cFunc(\grave{\mRat})
  \\ \Ex_{t}[\left(\cFunc^{\prime}(\grave{\mRat})(\mRat_{t+1}-\grave{\mRat})\right)\pShk_{t+1}] & \approx 0
  \\ \Ex_{t}[\mRat_{t+1}] & \approx \grave{\mRat} 
\end{align*}
but at $\mRat=\check{\mRat}$, $\Ex_{t}[\mRat_{t+1}]=\check{\mRat}\approx\grave{\mRat}$.
\end{comment}

\end{document}

% Local Variables:
% eval: (setq TeX-command-list  (assq-delete-all (car (assoc "BibTeX" TeX-command-list)) TeX-command-list))
% eval: (setq TeX-command-list  (assq-delete-all (car (assoc "BibTeX" TeX-command-list)) TeX-command-list))
% eval: (setq TeX-command-list  (assq-delete-all (car (assoc "BibTeX" TeX-command-list)) TeX-command-list))
% eval: (setq TeX-command-list  (assq-delete-all (car (assoc "BibTeX" TeX-command-list)) TeX-command-list))
% eval: (setq TeX-command-list  (assq-delete-all (car (assoc "Biber"  TeX-command-list)) TeX-command-list))
% eval: (add-to-list 'TeX-command-list '("BibTeX" "bibtex ../LaTeX/%s" TeX-run-BibTeX nil t                                                                              :help "Run BibTeX") t)
% eval: (add-to-list 'TeX-command-list '("BibTeX" "bibtex ../LaTeX/%s" TeX-run-BibTeX nil (plain-tex-mode latex-mode doctex-mode ams-tex-mode texinfo-mode context-mode) :help "Run BibTeX") t)
% TeX-PDF-mode: t
% TeX-file-line-error: t
% TeX-debug-warnings: t
% LaTeX-command-style: (("" "%(PDF)%(latex) %(file-line-error) %(extraopts) -output-directory=../LaTeX %S%(PDFout)"))
% TeX-source-correlate-mode: t
% TeX-parse-self: t
% eval: (cond ((string-equal system-type "darwin") (progn (setq TeX-view-program-list '(("Skim" "/Applications/Skim.app/Contents/SharedSupport/displayline -b %n ../LaTeX/%o %b"))))))
% TeX-parse-all-errors: t
% End:
