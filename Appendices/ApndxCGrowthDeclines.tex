\providecommand{\econtexRoot}{}
\renewcommand{\econtexRoot}{..}
\providecommand{\econtexPaths}{}\renewcommand{\econtexPaths}{\econtexRoot/Resources/econtexPaths}
% The \commands below are required to allow sharing of the same base code via Github between TeXLive on a local machine and Overleaf (which is a proxy for "a standard distribution of LaTeX").  This is an ugly solution to the requirement that custom LaTeX packages be accessible, and that Overleaf seems to ignore symbolic links (even if they are relative links to valid locations)
\providecommand{\econtex}{\econtexRoot/Resources/texmf-local/tex/latex/econtex}
\providecommand{\econtexSetup}{\econtexRoot/Resources/texmf-local/tex/latex/econtexSetup}
\providecommand{\econtexShortcuts}{\econtexRoot/Resources/texmf-local/tex/latex/econtexShortcuts}
\providecommand{\econtexBibMake}{\econtexRoot/Resources/texmf-local/tex/latex/econtexBibMake}
\providecommand{\econtexBibStyle}{\econtexRoot/Resources/texmf-local/bibtex/bst/econtex}
\providecommand{\econtexBib}{economics}
\providecommand{\notes}{\econtexRoot/Resources/texmf-local/tex/latex/handout}
\providecommand{\handoutSetup}{\econtexRoot/Resources/texmf-local/tex/latex/handoutSetup}
\providecommand{\handoutShortcuts}{\econtexRoot/Resources/texmf-local/tex/latex/handoutShortcuts}
\providecommand{\handoutBibMake}{\econtexRoot/Resources/texmf-local/tex/latex/handoutBibMake}
\providecommand{\handoutBibStyle}{\econtexRoot/Resources/texmf-local/bibtex/bst/handout}

\providecommand{\FigDir}{\econtexRoot/Figures}
\providecommand{\CodeDir}{\econtexRoot/Code}
\providecommand{\DataDir}{\econtexRoot/Data}
\providecommand{\SlideDir}{\econtexRoot/Slides}
\providecommand{\TableDir}{\econtexRoot/Tables}
\providecommand{\ApndxDir}{\econtexRoot/Appendices}

\providecommand{\ResourcesDir}{\econtexRoot/Resources}
\providecommand{\rootFromOut}{..} % Path back to root directory from output-directory
\providecommand{\LaTeXGenerated}{\econtexRoot/LaTeX} % Put generated files in subdirectory
%\providecommand{\EqDir}{\econtexRoot/Equations} % Put generated files in subdirectory
\providecommand{\EqDir}{Equations} % Put generated files in subdirectory
\providecommand{\econtexPaths}{\econtexRoot/Resources/econtexPaths}
\providecommand{\LaTeXInputs}{\econtexRoot/Resources/LaTeXInputs}
\providecommand{\LtxDir}{LaTeX/}

\documentclass[\econtexRoot/BufferStockTheory]{subfiles}
\providecommand{\econtexRoot}{}
\renewcommand{\econtexRoot}{..}
\providecommand{\econtexPaths}{}\renewcommand{\econtexPaths}{\econtexRoot/Resources/econtexPaths}
% The \commands below are required to allow sharing of the same base code via Github between TeXLive on a local machine and Overleaf (which is a proxy for "a standard distribution of LaTeX").  This is an ugly solution to the requirement that custom LaTeX packages be accessible, and that Overleaf seems to ignore symbolic links (even if they are relative links to valid locations)
\providecommand{\econtex}{\econtexRoot/Resources/texmf-local/tex/latex/econtex}
\providecommand{\econtexSetup}{\econtexRoot/Resources/texmf-local/tex/latex/econtexSetup}
\providecommand{\econtexShortcuts}{\econtexRoot/Resources/texmf-local/tex/latex/econtexShortcuts}
\providecommand{\econtexBibMake}{\econtexRoot/Resources/texmf-local/tex/latex/econtexBibMake}
\providecommand{\econtexBibStyle}{\econtexRoot/Resources/texmf-local/bibtex/bst/econtex}
\providecommand{\econtexBib}{economics}
\providecommand{\notes}{\econtexRoot/Resources/texmf-local/tex/latex/handout}
\providecommand{\handoutSetup}{\econtexRoot/Resources/texmf-local/tex/latex/handoutSetup}
\providecommand{\handoutShortcuts}{\econtexRoot/Resources/texmf-local/tex/latex/handoutShortcuts}
\providecommand{\handoutBibMake}{\econtexRoot/Resources/texmf-local/tex/latex/handoutBibMake}
\providecommand{\handoutBibStyle}{\econtexRoot/Resources/texmf-local/bibtex/bst/handout}

\providecommand{\FigDir}{\econtexRoot/Figures}
\providecommand{\CodeDir}{\econtexRoot/Code}
\providecommand{\DataDir}{\econtexRoot/Data}
\providecommand{\SlideDir}{\econtexRoot/Slides}
\providecommand{\TableDir}{\econtexRoot/Tables}
\providecommand{\ApndxDir}{\econtexRoot/Appendices}

\providecommand{\ResourcesDir}{\econtexRoot/Resources}
\providecommand{\rootFromOut}{..} % Path back to root directory from output-directory
\providecommand{\LaTeXGenerated}{\econtexRoot/LaTeX} % Put generated files in subdirectory
%\providecommand{\EqDir}{\econtexRoot/Equations} % Put generated files in subdirectory
\providecommand{\EqDir}{Equations} % Put generated files in subdirectory
\providecommand{\econtexPaths}{\econtexRoot/Resources/econtexPaths}
\providecommand{\LaTeXInputs}{\econtexRoot/Resources/LaTeXInputs}
\providecommand{\LtxDir}{LaTeX/}

\onlyinsubfile{% https://tex.stackexchange.com/questions/463699/proper-reference-numbers-with-subfiles
    \csname @ifpackageloaded\endcsname{xr-hyper}{%
      \externaldocument{\econtexRoot/BufferStockTheory}% xr-hyper in use; optional argument for url of main.pdf for hyperlinks
    }{%
      \externaldocument{\econtexRoot/BufferStockTheory}% xr in use
    }%
    \renewcommand\labelprefix{}%
    % Initialize the counters via the labels belonging to the main document:
    \setcounter{equation}{\numexpr\getrefnumber{\labelprefix eq:Dummy}\relax}% eq:Dummy is the last number used for an equation in the main text; start counting up from there
}


\onlyinsubfile{\externaldocument{BufferStockTheory}}
\begin{document}
\hypertarget{ApndxCGrowthDeclines}{}
\section{When Is Consumption Growth Declining in
  \texorpdfstring{$m$}{m}?}\label{sec:ApndxCGrowthDeclines}\label{subsec:dcgdxneg}

Figure~\ref{fig:cGroTargetFig} depicts the expected consumption growth factor as a strictly
declining function of the cash-on-hand ratio. To investigate this,
define
\begin{align*}
  \pmb{\Upsilon}(\mRat_{t})  & \equiv  \PGro_{t+1} \usual{\cFunc}(\Rnorm_{t+1}\aFunc(\mRat_{t})+\tShkAll_{t+1})/\usual{\cFunc}(\mRat_{t})  = \cLevBF_{t+1}/\cLevBF_{t}
\end{align*}
and the proposition in which we are interested is
\begin{align*}
  (d/d\mRat_{t})\Ex_{t}[\underbrace{\pmb{\Upsilon}(\mRat_{t})}_{\equiv \pmb{\Upsilon}_{t+1}}]  & < 0  
\end{align*}
or differentiating through the expectations operator, what we want is
\begin{equation}\begin{gathered}\begin{aligned}
  \Ex_{t}\left[\PGro_{t+1} \left(\frac{\usual{\cFunc}^{\prime}(\mRat_{t+1})\Rnorm_{t+1}\aFunc^{\prime}(\mRat_{t})\usual{\cFunc}(\mRat_{t})-\usual{\cFunc}(\mRat_{t+1})\usual{\cFunc}^{\prime}(\mRat_{t})}{\usual{\cFunc}{(\mRat_{t})}^{2}}\right)\right]  & < 0 \label{eq:kappaPrimeLT0}.
\end{aligned}\end{gathered}\end{equation}

Henceforth indicating appropriate arguments by the corresponding
subscript (e.g.\ $\cFunc_{t+1}^{\prime} \equiv \cFunc^{\prime}(\mRat_{t+1})$), since
$\PGro_{t+1}\Rnorm_{t+1}=\Rfree$, the portion of the LHS of equation~\eqref{eq:kappaPrimeLT0} in brackets can be manipulated to yield
\begin{equation}\begin{gathered}\begin{aligned}
  \cFunc_{t} \pmb{\Upsilon}^{\prime}_{t+1}  & = \cFunc^{\prime}_{t+1}\aFunc^{\prime}_{t}\Rfree-\cFunc^{\prime}_{t} \PGro_{t+1} \cFunc_{t+1}/\cFunc_{t} \nonumber
  \\  & = \cFunc^{\prime}_{t+1}\aFunc^{\prime}_{t}\Rfree-\cFunc^{\prime}_{t} \pmb{\Upsilon}_{t+1} \label{eq:cPrimek}
        .
\end{aligned}\end{gathered}\end{equation}

Now differentiate the Euler equation with respect to $\mRat_{t}$:
\begin{equation}\begin{gathered}\begin{aligned}
  1  & = \Rfree \DiscFac \Ex_{t}[ \pmb{\Upsilon}_{t+1}^{-\CRRA}] \notag
  \\ 0  & = \Ex_{t}[\pmb{\Upsilon}_{t+1}^{-\CRRA-1} \pmb{\Upsilon}_{t+1}^{\prime}] \notag
  \\  & = \Ex_{t}[\pmb{\Upsilon}_{t+1}^{-\CRRA-1}]\Ex_{t}[\pmb{\Upsilon}_{t+1}^{\prime}]+\mbox{cov}_{t}(\pmb{\Upsilon}_{t+1}^{-\CRRA-1},\pmb{\Upsilon}_{t+1}^{\prime}) \notag
  \\ \Ex_{t}[\pmb{\Upsilon}_{t+1}^{\prime}]  & = -\mbox{cov}_{t}(\pmb{\Upsilon}_{t+1}^{-\CRRA-1},\pmb{\Upsilon}_{t+1}^{\prime})/\Ex_{t}[\pmb{\Upsilon}_{t+1}^{-\CRRA-1}] \label{eq:covgen}
\end{aligned}\end{gathered}\end{equation}
but since $\pmb{\Upsilon}_{t+1} > 0$ we can see from~\eqref{eq:covgen} that~\eqref{eq:kappaPrimeLT0} is equivalent to
\begin{equation}\begin{gathered}\begin{aligned}
  \mbox{cov}_{t}(\pmb{\Upsilon}_{t+1}^{-\CRRA-1},\pmb{\Upsilon}_{t+1}^{\prime})  & > 0 \nonumber
\end{aligned}\end{gathered}\end{equation}
which, using~\eqref{eq:cPrimek}, will be true if
\begin{equation}\begin{gathered}\begin{aligned}
  \mbox{cov}_{t}(\pmb{\Upsilon}_{t+1}^{-\CRRA-1},\cFunc^{\prime}_{t+1}\aFunc^{\prime}_{t}\Rfree - \cFunc^{\prime}_{t}\pmb{\Upsilon}_{t+1})  & > 0 \notag
\end{aligned}\end{gathered}\end{equation}
which in turn will be true if both
\begin{equation}\begin{gathered}\begin{aligned}
  \mbox{cov}_{t}(\pmb{\Upsilon}_{t+1}^{-\CRRA-1},\cFunc^{\prime}_{t+1} )  & > 0 \notag
\end{aligned}\end{gathered}\end{equation}
and
\begin{align*}
  \mbox{cov}_{t}(\pmb{\Upsilon}_{t+1}^{-\CRRA-1},\pmb{\Upsilon}_{t+1})  & < 0. \notag
\end{align*}

The latter proposition is obviously true under our assumption $\CRRA > 1$.  The former will be true if
\begin{align*}
  \mbox{cov}_{t}\left({(\PGro \pShk_{t+1} \cFunc(\mRat_{t+1}))}^{-\CRRA-1},\cFunc^{\prime}(\mRat_{t+1}) \right)  & > 0 \nonumber.
\end{align*}

The two shocks cause two kinds of variation in $\mRat_{t+1}$.
Variations due to $\tShkAll_{t+1}$ satisfy the proposition, since a
higher draw of $\tShkAll$ both reduces $c_{t+1}^{-\CRRA-1}$ and
reduces the marginal propensity to consume.  However, permanent shocks
have conflicting effects.  On the one hand, a higher draw of
$\pShk_{t+1}$ will reduce $\mRat_{t+1}$, thus increasing both
$c_{t+1}^{-\CRRA-1}$ and $c_{t+1}^{\prime}$.  On the other hand, the
$c_{t+1}^{-\CRRA-1}$ term is multiplied by $\PGro \pShk_{t+1}$, so the
effect of a higher $\pShk_{t+1}$ could be to decrease the first term
in the covariance, leading to a negative covariance with the second
term.  (Analogously, a lower permanent shock $\pShk_{t+1}$ can also
lead a negative correlation.)

\end{document}
\endinput

% Local Variables:
% eval: (setq TeX-command-list  (assq-delete-all (car (assoc "BibTeX" TeX-command-list)) TeX-command-list))
% eval: (setq TeX-command-list  (assq-delete-all (car (assoc "BibTeX" TeX-command-list)) TeX-command-list))
% eval: (setq TeX-command-list  (assq-delete-all (car (assoc "BibTeX" TeX-command-list)) TeX-command-list))
% eval: (setq TeX-command-list  (assq-delete-all (car (assoc "BibTeX" TeX-command-list)) TeX-command-list))
% eval: (setq TeX-command-list  (assq-delete-all (car (assoc "Biber"  TeX-command-list)) TeX-command-list))
% eval: (add-to-list 'TeX-command-list '("BibTeX" "bibtex ../LaTeX/%s" TeX-run-BibTeX nil t                                                                              :help "Run BibTeX") t)
% eval: (add-to-list 'TeX-command-list '("BibTeX" "bibtex ../LaTeX/%s" TeX-run-BibTeX nil (plain-tex-mode latex-mode doctex-mode ams-tex-mode texinfo-mode context-mode) :help "Run BibTeX") t)
% TeX-PDF-mode: t
% TeX-file-line-error: t
% TeX-debug-warnings: t
% LaTeX-command-style: (("" "%(PDF)%(latex) %(file-line-error) %(extraopts) -output-directory=../LaTeX %S%(PDFout)"))
% TeX-source-correlate-mode: t
% TeX-parse-self: t
% eval: (cond ((string-equal system-type "darwin") (progn (setq TeX-view-program-list '(("Skim" "/Applications/Skim.app/Contents/SharedSupport/displayline -b %n ../LaTeX/%o %b"))))))
% TeX-parse-all-errors: t
% End:
