\providecommand{\econtexRoot}{}
\renewcommand{\econtexRoot}{..}
\providecommand{\econtexPaths}{}\renewcommand{\econtexPaths}{\econtexRoot/Resources/econtexPaths}
% The \commands below are required to allow sharing of the same base code via Github between TeXLive on a local machine and Overleaf (which is a proxy for "a standard distribution of LaTeX").  This is an ugly solution to the requirement that custom LaTeX packages be accessible, and that Overleaf seems to ignore symbolic links (even if they are relative links to valid locations)
\providecommand{\econtex}{\econtexRoot/Resources/texmf-local/tex/latex/econtex}
\providecommand{\econtexSetup}{\econtexRoot/Resources/texmf-local/tex/latex/econtexSetup}
\providecommand{\econtexShortcuts}{\econtexRoot/Resources/texmf-local/tex/latex/econtexShortcuts}
\providecommand{\econtexBibMake}{\econtexRoot/Resources/texmf-local/tex/latex/econtexBibMake}
\providecommand{\econtexBibStyle}{\econtexRoot/Resources/texmf-local/bibtex/bst/econtex}
\providecommand{\econtexBib}{economics}
\providecommand{\notes}{\econtexRoot/Resources/texmf-local/tex/latex/handout}
\providecommand{\handoutSetup}{\econtexRoot/Resources/texmf-local/tex/latex/handoutSetup}
\providecommand{\handoutShortcuts}{\econtexRoot/Resources/texmf-local/tex/latex/handoutShortcuts}
\providecommand{\handoutBibMake}{\econtexRoot/Resources/texmf-local/tex/latex/handoutBibMake}
\providecommand{\handoutBibStyle}{\econtexRoot/Resources/texmf-local/bibtex/bst/handout}

\providecommand{\FigDir}{\econtexRoot/Figures}
\providecommand{\CodeDir}{\econtexRoot/Code}
\providecommand{\DataDir}{\econtexRoot/Data}
\providecommand{\SlideDir}{\econtexRoot/Slides}
\providecommand{\TableDir}{\econtexRoot/Tables}
\providecommand{\ApndxDir}{\econtexRoot/Appendices}

\providecommand{\ResourcesDir}{\econtexRoot/Resources}
\providecommand{\rootFromOut}{..} % Path back to root directory from output-directory
\providecommand{\LaTeXGenerated}{\econtexRoot/LaTeX} % Put generated files in subdirectory
%\providecommand{\EqDir}{\econtexRoot/Equations} % Put generated files in subdirectory
\providecommand{\EqDir}{Equations} % Put generated files in subdirectory
\providecommand{\econtexPaths}{\econtexRoot/Resources/econtexPaths}
\providecommand{\LaTeXInputs}{\econtexRoot/Resources/LaTeXInputs}
\providecommand{\LtxDir}{LaTeX/}

\documentclass[\econtexRoot/BufferStockTeory.tex]{subfiles}
\begin{document}

\section{Equality of Aggregate Consumption Growth and Income Growth with Transitory Shocks}\label{sec:CGroEqPGro}

Section \ref{subsec:cGroEqPGroQ} asserted that in the absence of permanent shocks it is possible to prove
that the growth factor for aggregate consumption approaches that for aggregate permanent
income.  This section establishes that result.

Suppose the population starts in period $t$ with an arbitrary value for
 $\mbox{cov}_{t}(a_{t+1,i},\pLevBF_{t+1,i})$. 
  Then if $\mSS$ is the invariant mean
level of $\mRat$ we can define a `mean MPS away from $\mSS$' function
\begin{align}
 \acute{\aFunc}(\Delta)  & =  \Delta^{-1}\int_{\mSS}^{\mSS+\Delta} \aFunc^{\prime}(z)
 dz \label{eq:checkaFunc} \nonumber
\end{align}
and since $\pShk_{t+1,i}=1$, $\Rnorm_{t+1,i}$ is a constant at $\Rnorm$ we can write
\begin{align}
  a_{t+1,i} 
& =   \aFunc(\mSS)+(\mRat_{t+1,i}-\mSS)\acute{\aFunc}(\overbrace{\Rnorm
    a_{t,i}+\tShkAll_{t+1,i}}^{\mRat_{t+1,i}}-\mSS) \nonumber
\end{align}
so
\begin{align}
\mbox{cov}_{t}(a_{t+1,i},\pLevBF_{t+1,i})
 & = \mbox{cov}_{t}\left(\acute{\aFunc}(\Rnorm  a_{t,i}+\tShkAll_{t+1,i}-\mSS)
  ,\PGro   \pLevBF_{t,i}\right). \nonumber
\end{align}

But since ${\Rfree}^{-1}(\pZero  \Rfree\DiscFac)^{1/\CRRA} < \acute{\aFunc}(\mRat) < \PatR $,
\begin{equation}
  |\mbox{cov}_{t}((\pZero  \Rfree\DiscFac)^{1/\CRRA}a_{t+1,i},\pLevBF_{t+1,i})| <
  |\mbox{cov}_{t}(a_{t+1,i},\pLevBF_{t+1,i})| <
  |\mbox{cov}_{t}({\Pat}a_{t+1,i},\pLevBF_{t+1,i})| \nonumber
\end{equation}
and for the version of the model with no permanent shocks the \GICNrm~
says that
${\Pat} < \PGro, $ which implies
\begin{align}
  |\mbox{cov}_{t}(a_{t+1,i},\pLevBF_{t+1,i})| < \PGro
  |\mbox{cov}_{t}(a_{t,i},\pLevBF_{t,i})|. \nonumber
\end{align}


This means that from any arbitrary starting value, the relative
size of the covariance term shrinks to zero over time (compared
to the $\ASS \PGro^{n}$ term which is growing steadily
by the factor $\PGro$).  Thus, $\lim_{n \rightarrow \infty} \ALevBF_{t+n+1}/\ALevBF_{t+n} = \PGro$.

This logic unfortunately does not go through when there are permanent
shocks, because the $\Rnorm _{t+1,i}$ terms are not independent
of the permanent income shocks.

To see the problem clearly, define $\breve{\Rnorm }=\Mean\left[\Rnorm _{t+1,i}\right]$ and consider a first order Taylor expansion of $\acute{\aFunc}(\mRat_{t+1,i})$ around $\mTrg_{t+1,i}=\breve{\Rnorm } a_{t,i}+1,$
\begin{align*}
  \acute{\aFunc}_{t+1,i} & \approx  & %
  \acute{\aFunc}(\mTrg_{t+1,i})+\acute{\aFunc}^{\prime}(\mTrg_{t+1,i})\left(\mRat_{t+1,i}-\mTrg_{t+1,i}\right)
 \nonumber.
\end{align*}
%


The problem comes from the $\acute{\aFunc}^{\prime}$ term.  The
concavity of the consumption function implies convexity of the
$\aFunc$ function, so this term is strictly positive but we have no
theory to place bounds on its size as we do for its level $\acute{\aFunc}$.
We cannot rule out by theory that a positive shock to permanent income (which has a
negative effect on $\mRat_{t+1,i}$) could have an unboundedly positive
effect on $\acute{\aFunc}^{\prime}$ (as for instance if it pushes the
consumer arbitrarily close to the self-imposed liquidity constraint).

\end{document}
